\documentclass{article}
\usepackage[utf8]{inputenc}
\usepackage[T1]{fontenc}
\usepackage{url}
\usepackage{metalogo}

\newcommand*\lbd{\textsf{luabidi}}
\newcommand*\Lbd{\textsf{Luabidi}}
\newcommand*{\cmd}[1]{\texttt{\textbackslash #1}}

\def\fileversion{0.4}
\def\filedate{2019/08/24}

\usepackage[osf]{libertine}
\usepackage[scaled=.8]{beramono}

\begin{document}
	
\title{\lbd\\bidirectional typesetting in \LuaTeX}

\date{v.\,\fileversion, \filedate}

\author{Vafa Khalighi \and Arthur Reutenauer\thanks{%
		Current maintainer. Please submit bug reports and feature requests to \protect\url{https://github.com/bidi-tex/luabidi/issues}.}
	    \and Jürgen Spitzmüller}

\maketitle

\section{Objectives}
\Lbd\ is an attempt to provide bidirectional writing support for the \LuaTeX\ engine in the same vein as the \textsf{bidi} package\footnote{%
See \url{https://ctan.org/pkg/bidi}.} enables bidirectional writing with \XeTeX. The most prominent user of this package is \textsf{polyglossia}\footnote{%
See \url{https://ctan.org/pkg/polyglossia}.} which uses \lbd\ with RTL languages and \LuaTeX\ output (as opposed to \textsf{bidi} with \XeTeX). Note, however, that \lbd\ is much more limited than \textsf{bidi}. Currently, only a very basic
subset of the latter's features is supported.

\section{Package Options}

\begin{description}
	\item[arabmaths] By default, \lbd\ generates left-to-right maths. If you would like to have right-to-left
maths, use this option.
	\item[textwidthfootnoterule] expands the footnote to the whole text width.
% FIXME Does not seem to work yet.
%	\item[autofootnoterule] sets the footnote rule right or left aligned, depending on the direction of the first footnote
%	that follows the rule (i.\,e., that comes on the current page).
\end{description}

\section{User Commands}

\subsection{Main Text Direction}

By default, the main directionality of the document is left-to-right. To change it to right-to-left, use the switch

\begin{quote}
	\cmd{setRTLmain}
\end{quote}

\subsection{Paragraph Text Direction}

To change directionality for paragraphs, you can use the following switches:
\begin{description}
	\item[\cmd{setRTL}] (alias: \cmd{setRL}, \cmd{unsetLTR}) Change paragraph directionality to right-to-left.
	\item[\cmd{setLTR}] (alias: \cmd{setLR}, \cmd{unsetRTL}) Change paragraph directionality to left-to-right.
\end{description}
%
Alternatively, you can also use the environments

\begin{quote}
	\begin{verbatim}
	\begin{RTL}
		...
	\end{RTL}
	\end{verbatim}
\end{quote}
%
or

\begin{quote}
	\begin{verbatim}
	\begin{LTR}
	...
	\end{LTR}
	\end{verbatim}
\end{quote}
 
\subsection{Footnotes}

\Lbd\ provides an additional footnote command, \cmd{Footnote}, in addition to the standard \cmd{footnote} command.
The standard
\cmd{footnote} command thereby places the footnote always on the side that is currently the origin of direction:
on the left side of the page in LTR paragraphs and on the right in LTR paragraphs.

The uppercased \cmd{Footnote} command, in contrast, always places the footnote on the left side, notwithstanding the current
directionality. As opposed to \cmd{footnote}, \cmd{Footnote} does not have an optional argument to customize the number.

Consequently,
\begin{itemize}
	\item you should use \cmd{Footnote} if you want to have footnotes placed at the left in any case
	\item and \cmd{footnote} to get right-aligned footnotes in RTL context.
	\item At the moment, there is no command provided to set right-aligned footnotes in LTR paragraphs.
	This might be added in later releases.
\end{itemize}
%
The placement of the footnote rule depends on the main text directionality:

\begin{itemize}
	\item In the default mode (i.\,e., if \cmd{setRTLmain} is not used), the footnote rule is always set left-aligned (as usual in LTR documents).
	\item If the main direction is RTL (i.\,e., if \cmd{setRTLmain} is used), the footnote rule is always set right-aligned (as usual in LTR documents)
\end{itemize}



\section{Commands for Package Authors}

The following tests are provided to be used in packages:

\begin{description}
	\item[\cmd{if@RTL}] determines whether the current paragraph direction is right-to-left.
	\item[\cmd{if@RTLmain}] determines whether the main direction is right-to-left.
\end{description}
%
The following macros are provided
\begin{description}
	\item[\cmd{@ensure@RTL\{...\}}] Ensure that the argument is typeset RTL.
	\item[\cmd{@ensure@LTR\{...\}}] Ensure that the argument is typeset LTR.
	\item[\cmd{@ensure@dir\{...\}}] (alias: \cmd{@ensure@maindir\{...\}}) If used in RTL mode, the argument is put inside \cmd{RLE}, if used in LTR mode, the argument is output as is.
\end{description}

\section{Revision Log}

\begin{description}
	\item[v. 0.4 (2019/08/24)] Fix \cmd{@ensure@RTL}.
	\item[v. 0.3 (2019/07/10)] Fix compatibility with recent \LuaTeX\ (this version was never released to CTAN).
	\item[v. 0.2 (2013/05/27)] Fix additional files.
	\item[v. 0.1 (2009/04/01)] Initial release.
\end{description}

\end{document}